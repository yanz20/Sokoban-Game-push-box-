\documentclass[11pt, oneside]{article}
\usepackage{geometry} 
\geometry{letterpaper}
\usepackage{graphicx}
\usepackage{amssymb}
\usepackage{fancyhdr}
\pagestyle{fancy}
\lhead{
		Gurpartap Kaler,  kalerg1,
		400062310\\
		Freddie Yan,    yanz20,
		400079138\\
		Sagar Thomas,    thomas12,
		400054333}
\rhead{	Group 13\\
		\today}

\title{Problem Statement}
\author{Group 13}
%\date{}							% Activate to display a given date or no date
%\begin{header}

%\end{header}
\begin{document}
%\maketitle
\marginpar{}
\section*{Problem Statement}
\subsection*{Team Name:} The Box Group
\subsection*{Description of the issues that need to be addressed by your team:}

The purpose of the game, called Sokoban, is to push boxes and try to get them in their designated locations in a maze. The boxes are pushed by a character in the game that is controlled by the user. The designated locations are indicated by red squares in the maze. Some of the issues that need to be addressed by our team include: moving the character, creating the mazes, dealing with the design of the game and creating the proper logic of the game.
\subsection*{What problem are you trying to solve?}

Video games are usually created for entertainment purposes, without challenging your brain. Recreating Sokoban, a logical puzzle game, will keep people entertained and keep them challenged. This game requires much more thinking than your average video game, and we feel that the logic behind the puzzles will help keep users’ skills sharp. The original game was implemented in Java; however, we will be implementing it in Python and adding more features.

\subsection*{Why is this an important problem?}

Back in the day, Sokoban used to be pre-installed and played on the Nokia. Due to the increasing demand in smartphone, Nokia’s are no longer widely used. We would like to bring the same experience of Sokoban to the new generation. To do this, we plan to package the game into an app that can be installed and ran locally at any time on the user’s PC. We would also like to make Sokoban much more difficult by adding more challenging puzzles, to entertain­ the new generation.

\subsection*{What is the context of the problem you are solving?}

Sokoban is planned to be used by people of all ages, from youngsters to seniors. Our implementation will be available to be downloaded on all computers, regardless of the OS, including but not limited to Mac, Windows, and Linux. The game will be published for free and will be ran locally on the user’s computer. This is a very useful feature, as this generation is always on the move, making the game available offline. 

\end{document}  
