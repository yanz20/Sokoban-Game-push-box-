\documentclass[12pt, titlepage]{article}

\usepackage{fullpage}
\usepackage[round]{natbib}
\usepackage{multirow}
\usepackage{booktabs}
\usepackage{tabularx}
\usepackage{graphicx}
\usepackage{float}
\usepackage{hyperref}
\hypersetup{
    colorlinks,
    citecolor=black,
    filecolor=black,
    linkcolor=red,
    urlcolor=blue
}
\usepackage[utf8]{inputenc}


\newcounter{mnum}
\newcommand{\mthemnum}{M\themnum}
\newcommand{\mref}[1]{M\ref{#1}}

\title{Module Interface Specification}
\author{Group 13
        \\The Box Group
		\\ Gurpartap Kaler, kalerg1
		\\ Sagar Thomas, thomas12
		\\ Freddie Yan, yanz20}
\date{November 9, 2018}

\begin{document}

\maketitle
\pagenumbering{roman}
\tableofcontents

\newpage

\section{Module Hierarchy}

\begin{description}
    \item [\refstepcounter{mnum} \mthemnum \label{m1}:] Hardware-Hiding Module
    \item [\refstepcounter{mnum} \mthemnum \label{m2}:] Behaviour-Hiding Module: SceneManager Module
     \item [\refstepcounter{mnum} \mthemnum \label{m3}:] Behaviour-Hiding Module: Scene Module
    \item [\refstepcounter{mnum} \mthemnum \label{m4}:] Behaviour-Hiding Module: Character Module
    \item [\refstepcounter{mnum} \mthemnum \label{m5}:] Behaviour-Hiding Module: Box Module
    \item [\refstepcounter{mnum} \mthemnum \label{m6}:] Behaviour-Hiding Module: Game Module
\end{description}

\begin{table}[h!]
\centering
\begin{tabular}{p{0.3\textwidth} p{0.6\textwidth}}
\toprule
\textbf{Level 1} & \textbf{Level 2}\\
\midrule

{Hardware-Hiding Module} & ~ \\
\midrule

\multirow{4}{0.3\textwidth}{Behaviour-Hiding Module} & SceneManager Module \\
& Scene Module \\
& Character Module\\
& Box Module\\
& Game Module\\
\midrule

\multirow{1}{0.3\textwidth}{Software Decision Module} & N/A (No generic modules) \\
\bottomrule

\end{tabular}
\caption{Module Hierarchy}
\label{TblMH}
\end{table}

\section{MIS of Character Module}
\subsection{Interface Syntax}
\subsubsection{Exported Types}
KeyT = \{UP,DOWN,LEFT,RIGHT\}
\subsubsection{Exported Access Programs}
\begin{tabular}[pos]{|c|c|c|c|}
\hline
%	\label
\textbf{Name}& \textbf{In} & \textbf{Out} & \textbf{Exceptions} \\ \hline
 init&integer,integer,integer  &-  &invalid\_input \\ \hline
 move& KeyT & - &- \\ \hline
 getX& KeyT, integer & integer &- \\ \hline
 getY& KeyT, integer & integer &- \\ \hline
\end{tabular}
\subsection{Interface Semantics}
\subsubsection{State Variables}
x:integer//x-coordinate of the character\\
y:integer//y-coordinate of the character\\
speed:integer//length of a cell in maze
\subsubsection{Environmental Variables}
None
\subsubsection{Assumptions}
The constructor Character is called for each object instance before any other
access routine is called for that object.  The constructor cannot be called on
an existing object.

\subsubsection{Access Program Semantics}
init(i,j,s):\\
transition:x := i*s, y := j*s, speed := s.\\
exception:raise invalid\_argument if any of the inputs is not integer.\\
move(k):\\
transition:if k = UP: y:= y-speed, if k = DOWN: y:= y+speed,\\
if k = LEFT: x:= x-speed, if k = RIGHT: x:= x+speed.\\
exception:None.\\
getX(k,i):\\
Output:if k = UP: out:= x, if k = DOWN: out:= x,\\
if k = LEFT: out:=x-i*speed, if k = RIGHT: out:= x+i*speed.\\
exception:None.\\
getY(k,i):\\
Output:if k = UP: out:= y-i*speed, if k = DOWN: out:= y+i*speed,\\
if k = LEFT: out:=y, if k = RIGHT: out:= y.\\
exception:None.

\section{MIS of Box Module}
\subsection{Interface Syntax}
\subsubsection{Exported Types}
KeyT = \{UP,DOWN,LEFT,RIGHT\}
\subsubsection{Exported Access Programs}
\begin{tabular}[pos]{|c|c|c|c|}
\hline
%	\label
\textbf{Name}& \textbf{In} & \textbf{Out} & \textbf{Exceptions} \\ \hline
 init&integer,integer,integer  &-  &invalid\_input \\ \hline
 move& KeyT & - &- \\ \hline
 getX&  & integer &- \\ \hline
 getY&  & integer &- \\ \hline
\end{tabular}
\subsection{Interface Semantics}
\subsubsection{State Variables}
x:integer//x-coordinate of the Box\\
y:integer//y-coordinate of the Box\\
speed:integer//length of a cell in maze
\subsubsection{Environmental Variables}
None
\subsubsection{Assumptions}
The constructor Box is called for each object instance before any other
access routine is called for that object.  The constructor cannot be called on
an existing object.

\subsubsection{Access Program Semantics}
init(i,j,s):\\
transition:x := i*s, y := j*s, speed := s.\\
exception:raise invalid\_argument if any of the inputs is not integer.\\
move(k):\\
transition:if k = UP: y:= y-speed, if k = DOWN: y:= y+speed,\\
if k = LEFT: x:= x-speed, if k = RIGHT: x:= x+speed.\\
exception:None.\\
getX():\\
Output:out:= x.\\
exception:None.\\
getY():\\
Output:out:= y.\\
exception:None.



\section{MIS of Scene Module}
\subsection{Interface Syntax}
\subsubsection{Exported Types}
None
\subsubsection{Exported Access Programs}
\begin{tabular}[pos]{|c|c|c|c|}
\hline
%	\label
\textbf{Name}& \textbf{In} & \textbf{Out} & \textbf{Exceptions} \\ \hline
 init&Character,Sequence of Box, Sequence of integer  &-  & \\ \hline

\end{tabular}

\subsection{Interface Semantics}
\subsubsection{State Variables}
Char: Character//
Boxs:Sequence of Box//
maze:Sequence of integer
\subsubsection{Environmental Variables}
None
\subsubsection{Assumptions}
The constructor Scene is called for each object instance before any other
access routine is called for that object.  The constructor cannot be called on
an existing object.
\subsubsection{Access Program Semantics}
init(c,b,s):\\
transition: Char:= c, Boxs:= b, maze:= s.\\
exception:None

\section{MIS of SceneManager Module}
\subsection{Interface Syntax}
\subsubsection{Exported Access Programs}
\begin{tabular}[pos]{|c|c|c|c|}
\hline
%	\label
\textbf{Name}& \textbf{In} & \textbf{Out} & \textbf{Exceptions} \\ \hline
init &-  &-  &- \\ \hline
 setup\_scenes&Scene,Scene  & - &- \\ \hline
 update & Pygame.KeyEvent & - & -\\ \hline
 switch\_scene& -  & - & -\\ \hline
 get\_event& - & - & -\\ \hline
\end{tabular}
\subsection{Interface Semantics}
\subsubsection{State Variables}
scenes: Sequence of scenes // Contains all scenes in the game\\ 
done: Boolean // Contains the current running status of the game\\
scene\_name: String // Name of the current running scene\\
scene: Scene //The current running Scene\\
now: Integer // Current run-time of the game

\subsubsection{Environmental Variables}
Pygame.KeyEvent
\subsubsection{Assumptions}
The state variables will be assigned values before any scene management functions, such as switch\_scene are called
\subsubsection{Access Program Semantics} 

SceneManager(): \\
Input: none \\
Transition: Load up all the game scenes \\
Output: Assign initial values to state variables \\
Exceptions: None \\ 

setup\_scenes(scenes, start\_scene): \\
Input: scenes, start\_scene \\
Transition: Set the current scene to the start\_scene and store the rest of the scenes \\
Output: None \\
Exceptions: None \\

update(keys, now): \\
Input: Pygame.KeyEvent, now \\
Transition: Checks for key events that quit the game. If not, proceed to update the current scene. \\
Output: None \\
Exceptions: None \\

switch\_scene(): \\
Input: None \\
Transition: Changes the current scene to the next planned scene. \\
Output: New Scene \\
Exceptions: None \\

get\_event(event): \\
Input: None \\
Transition: Receives any event passed from the main game loop. Passes it to the current scene. \\
Output: None \\
Exceptions: None \\


\section{MIS of Game Module}
\subsection{Interface Syntax}
\subsubsection{Exported Access Programs}
\begin{tabular}[pos]{|c|c|c|c|}
\hline
%	\label
\textbf{Name}& \textbf{In} & \textbf{Out} & \textbf{Exceptions} \\ \hline
init & - & - & -\\ \hline
 update & - & - & -\\ \hline
 event\_loop& - & - & -\\ \hline
 run & - & - & -\\ \hline
 &  &  & \\ \hline
\end{tabular}
\subsection{Interface Semantics}
\subsubsection{State Variables}
now: Integer // The current run-time of the game \\
done: Boolean //Status of whether the game is complete or not \\
clock: Pygame.Clock //Clock object keeping track of fps \\
keys: Pygame.KeyEvent // Top level key events to be passed down to scenes\\
scene\_manager: SceneManager // Instance of the SceneManager module to manage all the games of the scene \\
screen: Pygame.Surface // Used for rendering game elements to the screen \\
\subsubsection{Environmental Variables}
Pygame.Clock \\
Pygame.KeyEvent \\
Pygame.Surface \\
\subsubsection{Assumptions}
None.
\subsubsection{Access Program Semantics}

Game(): \\
Input: None
Transition: Initializes all state variables to their default \\
Output: None \\
Exceptions: None \\

update(): \\
Input: None
Transition:  Updates the current run-time of the game\\
Output: None \\
Exceptions: None \\

event\_loop(): \\
Input: None
Transition:  Passes the Pygame events into scene\_manager \\
Output: None \\
Exceptions: None \\

run(): \\
Input: None
Transition:  Main game loop. Calls event\_loop and update repeatedly. \\
Output: None \\
Exceptions: None \\


\section{Major Revision History}
November 9, 2018 - Revision 0

\end{document}
